\documentclass[a4paper]{article}

%% Language and font encodings
\usepackage[english]{babel}
\usepackage[utf8x]{inputenc}
\usepackage[T1]{fontenc}

%% Sets page size and margins
\usepackage[a4paper,top=3cm,bottom=2cm,left=3cm,right=3cm,marginparwidth=1.75cm]{geometry}

%% Useful packages
\usepackage{amsmath}
\usepackage{graphicx}
\usepackage[colorinlistoftodos]{todonotes}
\usepackage[colorlinks=true, allcolors=blue]{hyperref}
\usepackage{listings}
\usepackage{color}

\definecolor{dkgreen}{rgb}{0,0.6,0}
\definecolor{gray}{rgb}{0.5,0.5,0.5}
\definecolor{mauve}{rgb}{0.58,0,0.82}

\title{DBS - Projekt: 2. Iteration - Datenimport}
\author{XXXXXXXXXXXXXXXXXXXXX}

\begin{document}
\maketitle

Quellcode auf: \href{https://github.com/0rC0/DBS-election/tree/dev/iteration2}{https://github.com/0rC0/DBS-election/tree/dev/iteration2}
\section{Aufgabe 1, 2 ,3}
\subsection{Beschreibung}

Alle aufgabe werden mit demselben Skript erledigt. Dies befindet sich auf GitHub auf am link \href{https://github.com/0rC0/DBS-election/blob/dev/iteration2/aufgabe_1_2_3.py}{https://github.com/0rC0/DBS-election/blob/dev/iteration2/aufgabe\_1\_2\_3.py }
Es wurden die folgenden Modulen benutzt:
\begin{verbatim}
import psycopg2 as pg2
import csv
import re
from datetime import datetime
\end{verbatim}
und die folgenden Hilfsfunktionen entwickelt:
\begin{verbatim}
def extract_hashtags(s):
    # Extract Hashtags from a string with regular expression and
    # return a list of those
    #source :https://stackoverflow.com/questions/2527892/parsing-a-tweet-to-extract-hashtags-into-an-array-in-python
    return re.findall(r"#(\w+)", s)


def string_to_bool(s):
    #Convert True and False from string to Boolean
    return s == 'True'


def make_timestamp(s):
    # Convert a string in a datetime object:
    # https://www.postgresql.org/docs/8.0/static/datatype-datetime.html
    return datetime.strptime(s, '%Y-%m-%dT%H:%M:%S')


def rm_bad_formatted_chars(s):
    # Encode in utf8 and eventually remove bad encoded characters
    try:
        output = s.encode('utf8')
    except:
        output = ''
        for char in s:
            try:
                char = char.encode('utf-8')
            except:
                char = ' '
            output += char
    return output


def mod_special_chars(s):
    # Characters like quote or semicolon can be annoing....
    return s.replace(';', ' ').replace("'", " ")
\end{verbatim}
Die Funktion main() führt die Aufgabe nacheinanders aus:
\begin{verbatim}
def main():
    cfg = {'host': '192.168.178.60',
           'user': 'elecuser',
           'pw': 'elecpass',
           'db': 'elections',
           'port': '5432'}
    aufgabe1(cfg)
    tweets, hashtags = aufgabe2()
    aufgabe3(cfg, tweets, hashtags)
\end{verbatim}

\subsection{Aufgabe 1}
Die .sql Detei ist an der link \href{https://github.com/0rC0/DBS-election/blob/dev/iteration2/DBSchema.sql}{https://github.com/0rC0/DBS-election/blob/dev/iteration2/DBSchema.sql}
\begin{verbatim}
def aufgabe1(cfg):
    # 1. Erstellen Sie in Ihrer Datenbank "Election" ein zu Ihrem relationalen
    # Modell passendes Datenbankschema mit allen Constraints.

    try:
        conn = pg2.connect(host = cfg['host'],
                            user = cfg['user'],
                            password = cfg['pw'],
                            database = cfg['db'],
                            sslmode = 'require',
                            port = cfg['port']) # Verbindung
        cur = conn.cursor() # Kursor
    except:
        print 'ConnectionProblem'

    # Erstelle die Relationen/Tabelle
    cur.execute('''
        CREATE TABLE tweets (
            handle           varchar(15) NOT NULL,
            text             varchar(500) NOT NULL,
            time             date NOT NULL,
            retweet_count    integer,
            favorite_count  integer,
            truncated        boolean,
            id               serial PRIMARY KEY);
        ''')
    cur.execute('''
        CREATE TABLE contains (
            hname varchar(140) NOT NULL,
            id integer NOT NULL);
        ''')
    cur.execute('''
        CREATE TABLE hashtags (
            hname varchar(140) NOT NULL);
        ''')
    conn.commit()
\end{verbatim}
\subsection{Aufgabe 2: Datenbereinigung}

Die Rohdatendatei enthält diversen Zeichnen mit unbekannter Codierung. Dies werden versucht in utf zu encoden, ansonsten mit Leerzeichnen ausgetauscht.
Die Daten werden auch angepasst, zum Beispiel, werden die boolesche werten ins Python boolsche Werte umgewandelt, sowie die Integers oder die Datetime, die als Strings importiert werden.
Während der Anpassung der Daten werden auch die Hashtags mithilfe eine Regular Expression extraiert und in ein Set hinzugefügt.
\begin{verbatim}
def aufgabe2():
    # Stellen Sie sicher, dass alle Daten, die Sie spaeter importieren werden
    # fehlerfrei sind. Reparieren oder verwerfen Sie betreffende Datensaetze.
    # Schreiben Sie ein Programm in einer Programmiersprache Ihrer Wahl,
    # dass diese Aufgabe fuer Sie uebernimmt.
    csv_file_name = 'american-election-tweets.csv'
    tweets = []
    hashtags = set()
    with open(csv_file_name, 'r') as f:
        datareader = csv.reader(f, delimiter=';')
        next(datareader)  # skip the header
        for row in datareader:
            # Format the data and put everything in a nice-looking JSON format
            d = dict()
            tweet_ht = extract_hashtags(row[1])
            for ht in tweet_ht:
                hashtags.add(ht)
            d['handle'] = row[0]
            d['text'] = mod_special_chars(rm_bad_formatted_chars(row[1]))
            d['hashtags'] = hashtags
            d['time'] = make_timestamp(row[4])
            d['retweet_count'] = int(row[7])
            d['favorite_count'] = int(row[8])
            d['truncated'] = string_to_bool(row[10])
            tweets.append(d)
    return tweets, hashtags
\end{verbatim}
\subsection{Aufgabe 3: Datenimport}
\begin{verbatim}
def aufgabe3(cfg, tweets, hashtags):
    # Importieren Sie die aufbereiteten Daten in Ihre Datenbank "Election".
    # Schreiben Sie hierzu ein Programm in einer Programmiersprache Ihrer
    # Wahl.
    try:
        conn = pg2.connect(host = cfg['host'],
                            user = cfg['user'],
                            password = cfg['pw'],
                            database = cfg['db'],
                            sslmode = 'require',
                            port = cfg['port']) # Verbindung
        cur = conn.cursor() # Kursor
    except:
        print 'ConnectionProblem'

    # Importiere die tweets
    for d in tweets:
        cur.execute('''
        INSERT INTO tweets(handle, text, time, retweet_count, favorite_count, truncated)
        VALUES ('{0}', '{1}', '{2}', '{3}', '{4}', '{5}')
        '''.format(d['handle'],
                   d['text'],
                   d['time'],
                   d['retweet_count'],
                   d['favorite_count'],
                   d['truncated']))
    conn.commit()

    # Importiere die Hashtags
    for ht in hashtags:
        cur.execute('''
        INSERT INTO hashtags(hname) VALUES ('{0}')
        '''.format(ht))
        cur.execute('''
        SELECT id FROM tweets WHERE text LIKE '%{0}%'
        '''.format(ht))
        ids = cur.fetchall() # touple of ids
        for i in ids:
            cur.execute('''
            INSERT INTO contains(hname, id) VALUES ('{0}', '{1}')
            '''.format(ht, i[0]))
    conn.commit()
\end{verbatim}

\section{4. Aufgabe: Webserver}
\begin{verbatim}
dbs@dbs:~$ sudo su
[sudo] Passwort für dbs: 
root@dbs:/home/dbs# apt-get install nginx-full
Paketlisten werden gelesen... Fertig
Abhängigkeitsbaum wird aufgebaut.       
Statusinformationen werden eingelesen.... Fertig
nginx-full ist schon die neueste Version (1.10.3-1ubuntu3).
Die folgenden Pakete wurden automatisch installiert und werden nicht mehr benötigt:
  linux-headers-4.10.0-19 linux-headers-4.10.0-19-generic linux-image-4.10.0-19-generic
  linux-image-extra-4.10.0-19-generic
Verwenden Sie »sudo apt autoremove«, um sie zu entfernen.
0 aktualisiert, 0 neu installiert, 0 zu entfernen und 21 nicht aktualisiert.
root@dbs:/home/dbs# cd /etc/nginx/sites-available/
root@dbs:/etc/nginx/sites-available# ls -la
insgesamt 16
drwxr-xr-x 2 root root 4096 Mai 17 20:39 .
drwxr-xr-x 8 root root 4096 Mai  2 17:56 ..
-rw-r--r-- 1 root root 2416 Feb 15 17:39 default
-rw-r--r-- 1 root root  611 Mai  2 19:26 uebung2
root@dbs:/etc/nginx/sites-available# cp default elections
root@dbs:/etc/nginx/sites-available# nano elections 
root@dbs:/etc/nginx/sites-available# cat elections 


# DBS-Election server configuration
#
server {
	listen 80;
	listen [::]:80;

	root /var/www/elections;

	# Add index.php to the list if you are using PHP
	index index.html index.htm index.nginx-debian.html;

	server_name _;

	location / {
		# First attempt to serve request as file, then
		# as directory, then fall back to displaying a 404.
		try_files $uri $uri/ =404;
	}
}
root@dbs:/etc/nginx/sites-available# rm ../sites-enabled/*
root@dbs:/etc/nginx/sites-available# ln -s /etc/nginx/sites-available/elections /etc/nginx/sites-enabled/elections
root@dbs:/etc/nginx/sites-available# mkdir /var/www/elections
root@dbs:/etc/nginx/sites-available# nano /var/www/elections/index.html
root@dbs:/etc/nginx/sites-available# service nginx reload
root@dbs:/etc/nginx/sites-available# curl localhost
<html>
<head>
<title>Elections</title>
</head>
<body>
</body>
</html>
root@dbs:/etc/nginx/sites-available# 

\end{verbatim}
\end{document}
